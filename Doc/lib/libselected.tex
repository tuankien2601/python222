% Copyright (c) 2005 Nokia Corporation
%
% Licensed under the Apache License, Version 2.0 (the "License");
% you may not use this file except in compliance with the License.
% You may obtain a copy of the License at
%
%     http://www.apache.org/licenses/LICENSE-2.0
%
% Unless required by applicable law or agreed to in writing, software
% distributed under the License is distributed on an "AS IS" BASIS,
% WITHOUT WARRANTIES OR CONDITIONS OF ANY KIND, either express or implied.
% See the License for the specific language governing permissions and
% limitations under the License.

\chapter{Selected Issues on Python Programming for S60}
\label{sec:selected}

The following issues must be considered when using Python on S60.

\section{Concurrency Aspects}
\label{subsec:concurrency}
The thread that initializes the Python interpreter becomes the main Python 
thread. This is usually the main thread of a UI application. When an 
application written in Python launches, the Symbian platform infrastructure 
creates the main UI thread that starts the Python environment. If a Python 
program is started as a server with \code{e32.start_server}, then the 
Python main thread is not a UI thread.

It is possible to launch new threads via the services of \module{thread} 
module. Examples of such situations could be to overcome eventual problems 
with the fixed, relatively small stack size of the main UI application 
thread; or to perform some background processing while still keeping the UI 
responsive. These new threads are not allowed to directly manipulate the UI; 
in other words, they may not use the \module{appuifw} module.

Because of the limitations of the Python interpreter's final cleanup, Python 
applications on the Symbian OS should be designed in such a way that the 
main thread is the last thread alive.

A facility called active object is used extensively on the Symbian OS to 
implement co-operative, non-preemptive scheduling within operating system 
threads. This facility is also utilized with native APIs. A Python 
programmer is exposed to related concurrency issues particularly in UI 
programming. Preserving the responsiveness of the UI with the help of active 
objects needs to be considered when designing the application logic. At the 
same time it is necessary to take into account the resulting concurrent 
behavior within the application when active objects are used. While the main 
execution path of a UI script is blocked in wait for an active object to 
complete -- either explicitly as a result of using \code{e32.Ao_lock}, 
or indirectly within some other Python API implementation -- the UI-related 
callbacks may still get called.

The standard \code{thread.lock} cannot normally be used for 
synchronization in the UI application main thread, as it blocks the UI event 
handling that takes place in the same thread context. The Symbian active 
object based synchronization service called \code{e32.Ao_lock} has been 
implemented to overcome this problem. The main thread can wait in this lock, 
while the UI remains responsive.

Python for S60 tries to minimize the unwanted exposure of a Python 
programmer to the active objects of the Symbian OS. The programmer may 
choose to implement the eventual concurrent behavior of the application with 
normal threads. However, certain active object based facilities are offered 
as an option in the \module{e32} module.

\section{Current S60 Python Script Execution Environment}
\label{subsec:current}

The current options for installing Python scripts to a S60 device are: a 
stand-alone installation to the device's main application menu, and an 
installation to a folder hierarchy maintained by the Python execution 
environment. For more details on this topic, see Programming with Python for 
S60 Platform \cite{PyS60Prog}. In the first case the script application is 
launched via application menu, and it executes in its own process context. The 
latter case is suitable for development, testing, and trying out new scripts.

The Python execution environment delivered with Python for S60 package 
has itself been written in Python. It is a collection of scripts that offer 
an interactive Python console and a possibility to execute scripts located 
in the directory of the execution environment. Due to this kind of design 
the scripts are not fully isolated from each other. This means that any 
changes a script makes in the shared execution environment are visible to 
other scripts as well. This may be helpful during the development of a 
script suite, as long as care is taken to avoid unwanted interference 
between scripts.

For some special issues to consider when writing Python scripts to be run from 
the current Python execution environment, see Programming with Python for S60 Platform \cite{PyS60Prog}. These include the arrangements for standard output 
and the maintenance of the Options menu contents.

\section{Standard I/O Streams}
\label{subsec:standard}

The standard Python I/O streams in the \module{sys} module are by default 
connected to underlying C STDLIB's \code{stdio} streams that in turn are 
terminated by dummy file descriptors. Usually Python scripts set the I/O 
streams suitably by manipulating them at Python level via \module{sys} 
module interface. The \module{e32} extension module offers a Python 
interface for attaching to C STDLIB's output streams, but this service is 
only recommended for debugging purposes. The \code{e32._stdo} function 
takes as its argument the name of the file where C STDLIB's \code{stdout} 
and \code{stderr} are to be redirected. This makes it possible to capture 
the low-level error output when the Python interpreter has detected a fatal 
error and aborts.

\section{Usage of Unicode}
\label{subsec:usage}
No changes have been made to the standard library modules with regard to 
string argument and return value types. S60 extensions generally 
accept both plain strings and Unicode strings as arguments, but they return 
only Unicode strings. APIs that take string arguments for the purpose of 
showing them on the UI expect Unicode strings. Giving something else may 
result in garbled appearance of the text on the screen.

\section{Date and Time}
\label{subsec:datetime}
Unix time, seconds since January 1, 1970, 00:00:00 UTC (Coordinated 
Universal Time), is generally used as the time format in the Python for 
S60 APIs described in this document. The float type is used for 
storing time values.

\section{Sharing Native Resources between Threads}
\label{subsec:sharing}

\begin{notice}[warning]
Python for S60 objects that wrap native resources cannot be shared
between threads. Trying this can lead to a crash. This is because
native resources cannot be shared between native threads. Examples:

\begin{itemize}
\item Symbian OS STDLIB implementation has some limitations that are reflected at OS module support (see S60 SDK documentation \cite{S60Doc}). For example, STDLIB file descriptors cannot be shared between threads, and for that reason, Python file objects cannot either. 
\item Sockets as implemented in the S60 version of the \module{socket} module.
\end{itemize}
\end{notice}

\section{Scalable User Interface}
\label{sec:scalable}

\begin{notice}[note]
S60 2nd Edition FP3 and further releases.
\end{notice}

S60 2nd Edition FP3 enables a new feature called scalable user interface. 
For Python developers scalable user interface is currently visible in new APIs 
supporting the scalable UI, icon loading, and new screen resolutions. For more 
information on scalable user interface, see Section \ref{subsec:icon}, Icon Type 
of this document, as well as Programming with Python for S60 Platform 
\cite{PyS60Prog}. 

\section{Error Handling}
\label{subsec:error}

The APIs described in this document may raise any standard Python 
exceptions. In situations where a Symbian error code is returned, its 
symbolic name is given as the value parameter of a \code{SymbianError} 
exception.

In case where the functions have nothing special to return, they return 
\code{None} on success.

\section{Limitations and Areas of Development}
\label{subsec:limitations}

Some OS level concepts to which the standard \module{os} library module 
offers an interface do not exist as such in Symbian OS environment. An 
example of this is the concept of current working directory.

Reference cycle garbage collection is not in use. Because of this, special 
care needs to be taken to dismantle cyclic references when a Python program 
exits. This prevents error messages related to native resources that are 
left open. The problem could be removed by developing support for collection 
of cyclic garbage or by performing a special cleanup action on interpreter 
exit. The \module{gc} module has been ported to the Symbian OS, and 
it has been verified to work. However, the current distribution has been 
built without \module{gc} support.
